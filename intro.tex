\spchapter{Введение}
При использовании любого устройства в космических приложениях~---~непосредственно в космических аппаратах или в наземном оборудовании~---~одной из ключевых проблем является обеспечение надежности работы устройства. Надежность работы требует тщательного контроля: контроля качества производства, контроля производительности в процессе работы, своевременной  диагностики и устранения возникающих неисправностей и т.д. Подобный контроль выполняется на основе информации, поступающей с датчиков, контролирующих работу устройства. Прогресс в развитии микроэлектроники за последние 10--15 лет привел к тому, что датчики стали существенно дешевле, легче и меньше по размерам. Это вызвало увеличение количества  используемых  датчиков  и  рост  объемов телеметрической  информации. Естественно, ручная обработка больших объемов  информации слишком трудоемка~–--~нужны средства автоматизации.

Задачи автоматизации поиска знаний решаются средствами интеллектуального анализа данных~--–~Data Mining. Фактически Data Mining~–--~это набор технологий поиска скрытых закономерностей в больших необработанных объемах данных. Data Mining является частью процесса KDD (Knowledge Discovering in databases), включающем, помимо поиска закономерностей, этапы сбора, подготовки данных и последующего анализа полученных результатов. К настоящему времени разработано множество алгоритмов и технологий Data Mining. Характерно, что универсального алгоритма для извлечения знаний из данных не существует. Каждое конкретное практическое приложение, обладающее специфическими характеристиками, требует либо адаптации технологий Data Mining, либо разработки новой технологии обработки данных.

Одним из ключевых направлений применения технологий Data Mining является автоматизация поиска аномалий. Поиск аномалий~---~это поиск шаблонов данных, не соответствующих ожидаемому поведению \cite{AnomalyDetectionASurvey}. Поиск аномалий широко применяется в задачах мониторинга состояния технический систем \cite{DerevyanenkoDataMining}. Для решения подобных задач используются системы ISHM (Integrated System Health Management). В ISHM состояние системы  контролируется по показаниям датчиков. Если в работе системы возникает  неисправность, в данных, поступающих с датчиков, возникают аномалии,  сигнализирующие об отклонении поведения системы от нормального поведения. Типичными задачами, решаемыми подобными системами мониторинга, являются определение факта возникновения аномалии, локализация ее местонахождения, диагностирование возникшей неисправности и прогнозирование возникновения неисправностей.

Традиционно системы ISHM используют одновременно несколько методов диагностики аномалий, в частности \cite{FaultDetectionByMiningAssocRules}:
\begin{itemize}
	\item проверку выхода значения параметра за установленные пределы;
	\item экспертную систему, содержащую набор правил, описывающих нормальное поведение системы (rule-based);
	\item математическую модель, описывающую требуемое поведение системы (model-based).
\end{itemize}

Общий  принцип у традиционных алгоритмов примерно один и тот же. Вначале эксперты задают модель поведения системы, представляющую набор правил,  характеризующих поведение системы. В процессе работы системы поступающие телеметрические данные проверяются на соответствие модели. Если поведение данных начинает отклоняться от модели, то оператору, контролирующему работу системы, поступает тревожный сигнал о возможной неисправности.

У всех традиционных алгоритмов есть общий недостаток~–--~они требуют  интенсивной работы экспертов. Эксперты задают набор правил, конструируют  математическую модель, устанавливают допустимые пределы значений параметров. Возрастает количество данных~–--~возрастает количество работы, которую необходимо проделать экспертам, прежде чем система мониторинга сможет работать.
 
Методы, основанные на Data Mining (data-­driven методы), от этого недостатка свободны. Data-­driven методы строят модель п-ведения системы автоматически,  на основе данных о нормальном поведении системы. Для обучения метода обычно достаточно несколько сотен точек нормальных данных.

Data-driven методы имеют ряд преимуществ по сравнению с традиционными:
\begin{itemize}
	\item не требуют априорно заданных знаний о работе системы;
	\item не требуют системного анализа, чтобы определить соотношения между параметрами;
	\item способны обрабатывать телеметрические данные,  поступающие от работающей системы, в режиме реального времени и очень быстро реагировать на появление аномалии; модель поведения системы очень компактна и позволяет вести работу в режиме реального времени;
	\item позволяют устанавливать и отслеживать взаимосвязь между большим количеством параметров;
	\item способны обнаруживать коллективные и контекстные аномалии \cite{AnomalyDetectionASurvey};
	\item дают возможность автоматически обрабатывать архивы накопленных данных и извлекать из них полезную информацию;
	\item позволяют легко учитывать новые данные о нормальном поведении системы и обновлять ранее построенную модель её поведения.
\end{itemize}