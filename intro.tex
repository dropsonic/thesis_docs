\spchapter{Введение}
Одной из ключевых проблем при эксплуатации летальных аппаратов (ЛА) является контроль и своевременная диагностика неисправностей. Подобный контроль выполняется на основе информации, поступающей с датчиков, контролирующих работу устройства. Для решения подобных задач используются системы ISHM (Integrated System Health Management) позволяющие оценить текущее и/или будущее состояние здоровья системы и интегрировать эту информацию в общую картину эксплуатационных потребностей с учётом имеющихся ресурсов~\cite{JennionsIVHM}. В ISHM состояние системы контролируется по показаниям датчиков. Прогресс в развитии микроэлектроники за последние 10--15 лет привел к тому, что датчики стали существенно дешевле, легче и меньше по размерам. Это вызвало увеличение количества используемых датчиков и рост объемов телеметрической информации. Естественно, ручная обработка больших объемов информации слишком трудоемка~–--~нужны средства автоматизации.

Традиционно системы ISHM используют одновременно несколько методов диагностики, в частности~\cite{FaultDetectionByMiningAssocRules}:
\begin{itemize}
	\item проверку выхода значения параметра за установленные пределы;
	\item экспертную систему, содержащую набор правил, описывающих нормальное поведение системы (rule-based);
	\item математическую модель, описывающую требуемое поведение системы (model-based).
\end{itemize}

Общий принцип у традиционных алгоритмов примерно один и тот же. Вначале эксперты задают модель поведения системы, представляющую набор правил, характеризующих поведение системы. В процессе работы системы поступающие телеметрические данные проверяются на соответствие модели. Если поведение данных начинает отклоняться от модели, то оператору, контролирующему работу системы, поступает тревожный сигнал о возможной неисправности.

У всех традиционных алгоритмов есть общий недостаток~–--~они требуют  интенсивной работы экспертов. Эксперты задают набор правил, конструируют  математическую модель, устанавливают допустимые пределы значений параметров. Возрастает количество данных~–--~возрастает количество работы, которую необходимо проделать экспертам, прежде чем система мониторинга сможет работать.

Данную задачу возможно автоматизировать средствами интеллектуального анализа данных~--–~Data Mining. Это собирательное название, используемое для обозначения совокупности методов обнаружения в данных ранее неизвестных, нетривиальных, практически полезных и доступных интерпретации знаний, необходимых для принятия решений в различных сферах человеческой деятельности~\cite{ShapiroDataMining}. Фактически Data Mining~---~это набор технологий поиска скрытых закономерностей в больших необработанных объемах данных. Data Mining является частью процесса KDD (Knowledge Discovering in Databases), включающем, помимо поиска закономерностей, этапы сбора, подготовки данных и последующего анализа полученных результатов. К настоящему времени разработано множество алгоритмов и технологий Data Mining. Характерно, что универсального алгоритма для извлечения знаний из данных не существует. Каждое конкретное практическое приложение, обладающее специфическими характеристиками, требует либо адаптации существующих методик Data Mining, либо разработки новой технологии обработки данных.

Одним из ключевых направлений применения технологий Data Mining является автоматизация поиска аномалий. Поиск аномалий~---~это поиск шаблонов данных, не соответствующих ожидаемому поведению~\cite{AnomalyDetectionASurvey}. Поиск аномалий широко применяется в задачах мониторинга состояния технический систем~\cite{DerevyanenkoDataMining}. Если в работе системы возникает неисправность, в данных, поступающих с датчиков, возникают аномалии, сигнализирующие об отклонении поведения системы от нормального поведения. Типичными задачами, решаемыми подобными системами мониторинга, являются определение факта возникновения аномалии, локализация ее местонахождения, диагностирование возникшей неисправности и прогнозирование возникновения неисправностей.

Методы диагностики аномалий, основанные на Data Mining (data-driven методы), свободны от недостатков традиционных методов и не требуют интенсивного участия экспертов для своей работы. Data-driven методы строят модель поведения системы автоматически на основе данных о нормальном поведении системы. Для обучения таким методам обычно достаточно несколько сотен точек нормальных данных.

Data-driven методы имеют ряд преимуществ по сравнению с традиционными:
\begin{itemize}
	\item не требуют априорно заданных знаний о работе системы;
	\item не требуют системного анализа, чтобы определить соотношения между параметрами;
	\item способны обрабатывать телеметрические данные, поступающие от работающей системы, в режиме реального времени и быстро реагировать на появление аномалии, т.к. модель поведения системы очень компактна;
	\item позволяют устанавливать и отслеживать взаимосвязь между большим количеством параметров;
	\item способны обнаруживать коллективные и контекстные аномалии~\cite{AnomalyDetectionASurvey};
	\item дают возможность автоматически обрабатывать архивы накопленных данных и извлекать из них полезную информацию;
	\item позволяют легко учитывать новые данные о нормальном поведении системы и обновлять ранее построенную модель её поведения.
\end{itemize}

\smallskip
Разработки систем мониторинга неисправностей на основе методов Data Mining активно ведутся в Японии~\cite{FaultDetectionByMiningAssocRules} и США~\cite{IversonGeneralPurposeDDSM, IversonISHM}. В последние годы за рубежом был разработан ряд data-driven методов и алгоритмов обнаружения аномалий, например, Orca, GritBot, IMS, GMM, LVS, одноклассовый SVM и др. Как показано в~\cite{MartinCompUnsupervisedDetectionMethods}, результаты работы разных методов могут отличаться, поэтому целесообразно их комбинировать.

Наиболее весомым доказательством эффективности ISHM-систем на основе данных методов в аэрокосмической отрасли является их успешное применение в NASA для диагностики неисправностей в ЛА типа «Шаттл» и их преемниках~---~серии «Ares»~\cite{IversonGeneralPurposeDDSM}. Пробный пуск системы на архивных данных показал, что установка такой системы на аппарате «Колумбия» серии «Шаттл» позволила бы избежать взрыва ЛА при посадке, повлёкшего гибель всего экипажа. Как известно, «Колумбия» потерпела катастрофу из-за отрыва куска изоляционной обшивки, пробившей термоизоляцию на левом крыле. Отрыв произошел во время старта корабля, однако о проблемах с термоизоляцией стало известно лишь через 17 дней, во время приземления шатла~\cite{ColumbiaAccidentReport}. База знаний ISHM строилась на основе анализа данных предыдущих 5 полетов «Колумбии». ISHM выдала сигнал о возникновении неисправности в течении двух минут с момента ее возникновения~\cite{IversonISHM, DerevyanenkoDataMining}. В данной системе совместно используются методы Orca и IMS~\cite{IversonSHMforSpaceMissionOperations}.

Подобные системы нашли применение на Международной Космической Станции (МКС) для контроля работоспособности и определения сроков ремонта и замены гиродинов (гиродин, англ. control moment gyroscope, сокр. CMG~---~вращающееся инерциальное устройство, применяемое для высокоточной ориентации и стабилизации, как правило, космических аппаратов (КА), обеспечивающее правильную ориентацию в полете и предотвращающее беспорядочное вращение~\cite{WikiGirodyn}). C 2008 года NASA ведёт работы по применению данных методов для контроля и диагностики других подсистем МКС~\cite{IversonSHMforSpaceMissionOperations}.

Japan Aerospace Exploration Agency (Японское агентство аэрокосмических исследований) с 2011 года ведёт разработку систем мониторинга состояния спутников на основе данных телеметрии. Основным используемым методом в данной системе является SVM~\cite{SVMSatelliteMonitoring}.