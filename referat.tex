\protect\spchapter*{\MakeUppercase{Реферат}}
\sloppy
{
\theauthor\ \MakeUppercase{\thethesistitle}, дипломная работа: \pagestotal~с., \figurestotal~рис., \tablestotal~табл., \bibitemstotal~ист., \appendicestotal~прил.

Ключевые слова: \MakeUppercase{\thekeywords}
}
\medskip

В данной дипломной работе разработан и реализован в виде программной системы метод мониторинга состояния ЛА на основе алгоритмов интеллектуального анализа данных (Data Mining).

Разработанный в работе метод позволяет производить контроль и диагностику состояния системы в реальном времени на основе архивных данных телеметрии при различных режимах работы, без априорных знаний о предметной области, назначении системы, её составе, конструкции. Метод относится к классу искусственного интеллекта, используя машинное обучение (обучение без учителя) для построения модели системы.

Программная реализация метода прошла тестирование, подтвердив свою эффективность с точки зрения быстродействия и достоверности контроля, позволяя своевременно обнаруживать аномалии в работе систем.

Работа имеет практическое значение в рамках деятельности научно-измерительных пунктов (НИП), где может применяться для своевременного контроля состояния ЛА и обнаружения аномалий в работе оборудования на основе данных телеметрии.