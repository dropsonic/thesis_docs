\newcommand{\thesistitle}{Разработка системы мониторинга состояния ЛА (Integrated System Health Management) на основе методов интеллектуального анализа данных (Data Mining)}
\newcommand{\thesisauthor}{Панченко В.В.}
\newcommand{\thesiskeywords}{контроль и диагностика, обнаружение аномалий, интеллектуальный анализ данных, Data Mining, кластеризация, Integrated System Health Monitoring, телеметрия}

\protect\spchapter*{\MakeUppercase{Реферат}}
\sloppy
{
\thesisauthor\ \MakeUppercase{\thesistitle}, дипломная работа: \pagecount~с., \total{figurecount}~рис., \total{tablecount}~табл., \total{refcount}~ист., \total{appendixcount}~прил.

Ключевые слова: \MakeUppercase{\thesiskeywords}
}
\medskip

В данной дипломной работе разработан и реализован в виде программной системы метод мониторинга состояния ЛА на основе алгоритмов интеллектуального анализа данных (Data Mining).

Разработанный в работе метод позволяет производить контроль и диагностику состояния системы в реальном времени на основе архивных данных телеметрии при различных режимах работы, без априорных знаний о предметной области, назначении системы, её составе, конструкции. Относится к классу искусственного интеллекта, используя машинное обучение (обучение без учителя) для построения модели системы.

Программная реализация метода прошла тестирование, подтвердив свою эффективность с точки зрения быстродействия и точности контроля, позволяя своевременно обнаруживать аномалии в работе систем.

Работа имеет практическое значение в рамках деятельности центров управления полётами.