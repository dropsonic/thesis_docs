\chapter{Специальная часть}
\section{Постановка задачи}
Разработать метод мониторинга состояния ЛА на основе методов интеллектуального анализа данных. Реализовать программную систему, использующую данный метод.

Система должна удовлетворять следующим требованиям:
\begin{itemize}
	\item строить модель системы только на основе телеметрии при различных режимах её работы, без априорных данных о её назначении, составе, конструкции (обучение без учителя);
	\item обладать способностью классифицировать аномалии в работе системы;
	\item обрабатывать большие массивы входных данных (несколько десятков тысяч точек) за конечное время;
	\item определять состояние системы в режиме реального времени.
\end{itemize}

%------------------------------------------------------------------
\section{Анализ существующих методов выявления аномалий без учителя}
На данный момент существует несколько методов, для которых доказана возможность применения их в системах контроля и диагностики ЛА. Такими методами являются Orca, IMS (Inductive Monitoring System), GritBot, GMM (Gaussian Mixture Model), LDS (Linear Dynamic System) и One-Class SVM (Support Vector Machine)~\cite{MartinCompUnsupervisedDetectionMethods}.

\subsection{Orca}
Orca~---~метод поиска аномалий без учителя, использующий подход «ближайшего соседа» (nearest neighbor) для поиска аномалий~\cite{SchwabacherMachLearnAppl}. Данный метод был разработан Стефеном Бэйем (Institute for the Study of Learning and Expertise) и Марком Швабахером (NASA Ames Research Center) и подробно описан в~\cite{BaySchwabacherOrca}. Orca относится к методам обнаружения аномалий, основанных на измерении расстояний между точками (distance-based).

Понятие аномалии для данного класса методов определено следующим образом: «объект $O$ в выборке $T$ является аномалией, если по крайней мере доля $p$ из всех объектов в $T$ лежит дальше от $O$, чем расстояние $D$»~\cite{KnorrNgDistBasedAlgorithms}. Distance-based методы являются обобщением некоторых статистических тестов на аномальность. Данный класс методов не требует априорных знаний о виде распределения для выборки. Кнорр и Нг предложили простейший алгоритм на вложенных циклах (Nested Loop, NL)~\cite{KnorrNgDistBasedAlgorithms}, который находит аномалии путём вычисления расстояния между всеми точками в исходной выборке. Сложность данного алгоритма составляет $O(kN^2)$, где $k$ --- размерность пространства, а $N$ --- размер выборки.

Несмотря на то, что были разработаны более эффективные c т.з. вычислительной сложности алгоритмы (\cite{TaoMiningDistBasedOutliersFromLargeDB}~и~\cite{AngiulliVeryEfficientMiningDistBasedOutliers}), на практике наиболее сложным является определение расстояния $D$, по достижению которого точку следует считать аномалией. Может потребоваться непредсказуемо большое число итераций, чтобы найти подходящее значение $D$. Найти интервал $[D_{min}, D_{max}]$ возможно путём полного перебора, как показано в~\cite{TaoMiningDistBasedOutliersFromLargeDB}, но данный подход обладает слишком высокой вычислительной сложностью.

В качестве решения данной проблемы было предложено следующее определение аномалии, не требующее задания $D$: «объект считается аномалией, если это один из $n$ объектов с наибольшим расстоянием до их $k$-ых ближайших соседей, где~$k,n\in\mathbb{N}$»~\cite{RamaswamyEffAlgoMiningOutliers}. Пользователю достаточно указать количество аномалий, которое должен вернуть алгоритм, без прямого указания дистанции $D$. Более того, возвращаемые алгоритмом аномалии будут ранжированы по степени аномальности, являющейся численной характеристикой.

Orca использует данный подход, развивая идею алгоритма на вложенных циклах~(NL). Данный алгоритм на больших массивах данных показывает сложность, близкую к линейной~\cite{BaySchwabacherOrca}.

Псевдокод алгоритма приведён в листинге~\ref{lst:spec:OrcaPseudocode}. Ключевыми особенностями алгоритма являются:
\begin{itemize}
	\item необходимость рандомизации исходных данных (строка~\ref{lst:spec:OrcaPseudocode:Random}). Для эффективной работы алгоритма требуется, чтобы объекты в выборке находились в случайном порядке. При обработке выборки на ПЗУ возможно рандомизировать выборку за линейное время и используя конечный объём памяти~\cite{BaySchwabacherOrca};
	\item использование вложенных циклов (строка~\ref{lst:spec:OrcaPseudocode:NL}). Основной идеей является отслеживание ближайших соседей для каждого объекта в $D$;
	\item правило отсечения (строка~\ref{lst:spec:OrcaPseudocode:Pruning}). Когда для ближайших соседей объекта степень аномальности становится меньше, чем величина среза, алгоритм удаляет данный объект, так как больше нет оснований считать его аномальным. Чем больше объектов перебирает алгоритм, тем выше становится величина среза, улучшая таким образом эффективность алгоритма по времени.
\end{itemize}

\begin{algorithm}[h]
\caption{Псевдокод алгоритма Orca}
\label{lst:spec:OrcaPseudocode}
\begin{algorithmic}[1]
\REQUIRE $k$, количество ближайших соседей; $n$, количество аномалий; $D$, выборка.
\ENSURE $O$, множество аномалий
\STATE Перемешать все объекты в выборке $D$. \label{lst:spec:OrcaPseudocode:Random}
\STATE Инициализировать величину среза нулём.
\WHILE{в выборке $D$ остались необработанные объекты}
	\STATE Загрузить фиксированное количество объектов $B$ в буфер.
	\FOR{каждого объекта $d$ в $D$} \label{lst:spec:OrcaPseudocode:NL}
		\FOR{каждого объекта $b$ в $B$}
			\STATE Вычислить расстояние между $b$ и $d$.
			\IF{$d$ ближе к $b$, чем $k$ ближайших соседей $b$}
				\STATE Заменить соседа с наибольшим расстоянием на $d$.
				\STATE Вычислить степень аномальности $b$.
				\IF{степень аномальности ниже величины среза} \label{lst:spec:OrcaPseudocode:Pruning}
					\STATE Удалить $b$ из $B$.
				\ENDIF
			\ENDIF
		\ENDFOR
	\ENDFOR
	\STATE Поместить в $O$ оставшиеся в $B$ объекты.
	\STATE Отсортировать объекты в $O$ по степени аномальности.
	\STATE Оставить в $O$ только $n$ объектов.
	\STATE Обновить величину среза степенью аномальности последнего объекта в $O$.
\ENDWHILE
\RETURN $O$.
\end{algorithmic}
\end{algorithm}

Преимуществами метода являются:
\begin{itemize}
	\item превосходная масштабируемость: на выборках большого объёма производительность алгоритма близка к линейной;
	\item низкие требования к памяти: не требуется загружать в память всю выборку;
	\item возможность задать любую метрику для расстояния и функцию для определения степени аномальности.
\end{itemize}

Недостатки следуют из природы метода. В качестве основных можно выделить следующие:
\begin{itemize}
	\item в худшем случае (например, когда выборка не содержит аномалий) производительность алгоритма крайне низкая. Из-за вложенных циклов может потребоваться $O(N^2)$ операций вычисления расстояния и $O(N/l \cdot N)$ операций доступа к данным, где $l$ --- размер буфера;
	\item в качестве результата алгоритм возвращает фиксированное число аномалий, указанное перед началом работы.
\end{itemize}

\subsection{GritBot}
К основным недостаткам относится тот факт, что данный метод загружает весь массив исходных данных в память~\cite{BaySchwabacherOrca}; таким образом, с его помощью невозможно обрабатывать сколь-либо большие выборки.