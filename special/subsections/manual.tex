\subsection{Руководство пользователя}
\subsubsection{Установка программного обеспечения}
Для работы программного обеспечения требуется ПЭВМ с установленной ОС Windows 7 и выше и наличием .NET Framework 4.5. Программное обеспечение представлено в виде нескольких файлов динамических библиотек (DLL) и файла приложения (\textit{Thesis.App.exe}). Для установки программы на компьютер достаточно скопировать данные файлы в любую папку на ПЗУ. Имя и расположение папки не имеет значения. На этом установку системы можно считать законченной.

\subsubsection{Формат входных данных}
Для работы программы требуется минимум два файла: файл с описанием параметров векторов и файл с обучающей выборкой для номинального режима работы ОК.

Во всех файлах существует возможность оставлять однострочные комментарии. Комментарий начинается с символа „\%“ и продолжается до конца строки. При чтении файлов программой комментарии игнорируются.

Файл с описанием параметров должен заполняться следующим образом. Каждый параметр представляется файле отдельной строкой, которая имеет формат:

\textsf{[весовой коэффициент] : имя : описание\_параметра}

Разделителями в данном файле служат двоеточие, точка с запятой и запятая. Весовой коэффициент является необязательным параметром.

Если параметр не должен учитываться программой, указывается ключевое слово \textit{ignorefeature}.

Пример: \textsf{time : ignorefeature}

Поддерживаются следующие типы параметров:
\begin{itemize}
	\item непрерывные (ключевое слово \textit{continuous}). \\ Пример: \textsf{pressure : continuous}
	\item дискретные с фиксированным множеством значений. Для таких параметров все возможные значения указываются после имени. \\ Пример: \textsf{state : opened, closed}
	\item дискретные с открытым множеством значений (ключевое слово \textit{discrete}). Для таких параметров система сама определяет возможные значения на основе данных из обучающих выборок. \\ Пример: \textsf{orbit : discrete}
\end{itemize}

Файлы с данными обучающих выборок представляют собой набор строк (записей), каждая из которых представляет собой вектор с набором значений. Значения в строке разделяются запятыми или точками с запятой. Количество, состав и тип значений должны строго соответствовать указанным в файле с описанием параметров. Если значение какого-либо параметра в данной записи отсутствует, вместо него ставится знак вопроса (по умолчанию) либо строка, указанная при запуске программы в качестве соответствующего параметра.

Пример записи:

\textsf{16:39:50, 45.5, opened, ?, high}

Имя файла (без расширения) для каждой обучающей выборки программа воспринимает как название соответствующего режима. Например, файл \textit{Nominal.txt} содержит обучающую выборку для режима «Nominal».

\subsubsection{Запуск программного обеспечения}
Исполняемым файлом программы является файл \textit{Thesis.App.exe}. Программа при запуске принимает на вход параметры, указанные в таблице~\ref{tab:spec:AppOptions} приложения~\ref{app:AppOptions}.

Пример команды запуска:

\texttt{thesis.app fields.txt -r nominal1.txt nominal2.txt [-a regime1.txt regimeN.txt] [-f threshold 0.5] [-d kmeans] [-m sqreuclid] [-n standard] [-v N/A]}

Если программа запущена без параметров либо параметры указаны неверно, то будет выведена справочная информация, как показано на рисунке~\ref{fig:spec:scr:AppHelp}.

\begin{figure}[h]
\includegraphics[width=0.7\textwidth]{scr_app_help}
\caption{Экран со справочной информацией}
\label{fig:spec:scr:AppHelp}
\end{figure}

\subsubsection{Работа с программным обеспечением}
Если все параметры для запуска были указаны корректно, программа запрашивает у пользователя ввод порогового значения $\varepsilon$, как показано на рисунке~\ref{fig:spec:scr:EnterEpsilon}.

\begin{figure}[h]
\includegraphics[width=0.7\textwidth]{scr_app_epsilon}
\caption{Ввод порогового значения в программу}
\label{fig:spec:scr:EnterEpsilon}
\end{figure}

После ввода программа отфильтровывает аномалии в указанных обучающих выборках и строит модель ОК. Сначала выводятся найденные аномалии для указанных обучающих выборок номинальных режимов, а затем построенная база кластеров для каждого режима работы (рисунок~\ref{fig:spec:scr:SystemModel}).

\begin{figure}[h]
\includegraphics[width=0.7\textwidth]{scr_app_systemmodel}
\caption{Построенная программой модель ОК}
\label{fig:spec:scr:SystemModel}
\end{figure}

После чего программа готова к мониторингу поступающих телеметрических данных, о чём сообщает пользователю соответствующим сообщением (рисунок~\ref{fig:spec:scr:EnterRecord}). Для выхода из режима мониторинга необходимо нажать \textit{Enter}.

\begin{figure}[h]
\includegraphics[width=0.7\textwidth]{scr_app_enterrecord}
\caption{Режим мониторинга: ввод записи}
\label{fig:spec:scr:EnterRecord}
\end{figure}

Записи вводятся в том же формате, что и во входных файлах. После ввода записи программа показывает результат мониторинга (текущий режим/текущий режим и расстояние до него/факт наличия аномалии, ближайший режим и расстояние до него). Пример работы программы в режиме мониторинга приведён на рисунке~\ref{fig:spec:scr:Monitoring}.

\begin{figure}[h]
\includegraphics[width=0.7\textwidth]{scr_app_monitoring}
\caption{Режим мониторинга}
\label{fig:spec:scr:Monitoring}
\end{figure}