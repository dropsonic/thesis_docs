\subsection{Анализ языков программирования}
Наиболее распространёнными и набирающими популярность языками программирования, обеспечивающими полноценную поддержку ООП и позволяющими разрабатывать кроссплатформенные приложения для настольных ПЭВМ, являются следующие~\cite{TIOBELanguageIndex}:
\begin{itemize}
	\item C++;
	\item C\#;
	\item D;
	\item Java;
	\item Python;
	\item Ruby;
	\item Visual Basic .NET.
\end{itemize}

Рассмотрим каждый из них более подробно.

\subsubsection{C++}
С++~---~компилируемый статически типизированный язык программирования общего назначения. Поддерживает такие парадигмы программирования как процедурное программирование, объектно-ориентированное программирование, обобщённое программирование, обеспечивает модульность, раздельную компиляцию, обработку исключений, абстракцию данных, объявление типов (классов) объектов, виртуальные функции. Стандартная библиотека включает, в том числе, общеупотребительные контейнеры и алгоритмы.

Язык возник в начале 1980-х годов как разработка сотрудника Bell Labs Бьёрна Страуструпа. 

C++ широко используется для разработки программного обеспечения, являясь одним из самых популярных языков программирования~\cite{TIOBELanguageIndex}. Область его применения включает создание операционных систем, разнообразных прикладных программ, драйверов устройств, приложений для встраиваемых систем, высокопроизводительных серверов, а также развлекательных приложений (игр). Существует множество реализаций языка C++, как бесплатных, так и коммерческих и для различных платформ. Например, на платформе x86 это GCC, Visual C++, Intel C++ Compiler, Embarcadero (Borland) C++ Builder и другие. C++ оказал огромное влияние на другие языки программирования, в первую очередь на Java и C\#.

Преимущества:
\begin{itemize}
	\item достаточно высокая производительность. Язык спроектирован так, чтобы дать программисту максимальный контроль над всеми аспектами структуры и порядка исполнения программы. Имеется возможность работы с памятью на низком уровне.
	\item мультипарадигменность языка и широчайший спектр возможностей;
	\item огромное количество готовых библиотек и расширений.
\end{itemize}

Недостатки:
\begin{itemize}
	\item плохо продуманный синтаксис;
	\item унаследованные от языка Си низкоуровневые свойства существенно тормозят и затрудняют прикладную разработку;
	\item язык содержит опасные возможности, существенно снижающие качество и надёжность программ сразу по всем показателям.~\cite{WikiCpp}
\end{itemize}

\subsubsection{C\#}
C\#~---~мультипарадигменный язык программирования. Разработан в 1998—2001 годах группой инженеров под руководством Андерса Хейлсберга в компании Microsoft как язык разработки приложений для платформы Microsoft .NET Framework и впоследствии был стандартизирован как ECMA-334 и ISO/IEC 23270.

C\# относится к семье языков с C-подобным синтаксисом. Язык имеет поддержку статической и динамической типизации, поддерживает полиморфизм, перегрузку операторов (в том числе операторов явного и неявного приведения типа), делегаты, атрибуты, события, свойства, обобщённые типы и методы, итераторы, анонимные функции с поддержкой замыканий, LINQ, исключения, комментарии в формате XML.

Переняв многое от своих предшественников~---~языков C++, Pascal, Модула, Smalltalk и в особенности Java~---~С\#, опираясь на практику их использования, исключает некоторые модели, зарекомендовавшие себя как проблематичные при разработке программных систем, например, C\#, в отличие от C++, не поддерживает множественное наследование классов (между тем допускается множественное наследование интерфейсов).

Код на языке C\# транслируется в код на промежуточном языке MSIL (Microsoft Intermediate Language), который исполняется в виртуальной машине. Существует несколько реализаций для разных платформ. Хотя он и предназначен для генерации кода, исполняемого в среде .NET, сам по себе он не является частью .NET. Однако поскольку язык C\# предназначен для применения на платформе .NET, то разработчику, важно иметь представление о .NET Framework, если он хочет эффективно разрабатывать приложения на С\#. 

Центральной частью каркаса .NET является его Общеязыковая исполняющая среда – Common Language Runtime (CLR), или .NET runtime. Код, исполняемый под управлением CLR, часто называют управляемым кодом. Однако перед тем как код сможет исполниться CLR, любой исходный текст (на C\# или другом языке) должен быть скомпилирован. Компиляция в .NET состоит из двух шагов: 
\begin{itemize}
	\item компиляция исходного кода в IL; 
	\item компиляция IL в специфический для платформы код с помощью CLR.
\end{itemize}
 
Этот двухшаговый процесс компиляции очень важен, потому что наличие IL (управляемого кода)~---~это ключ ко многим преимуществам .NET.

К преимуществам .NET следует отнести наличие промежуточного языка Microsoft (MSIL).  MSIL разделяет с псевдокодом Java идею низкоуровневого языка с простым синтаксисом (базирующегося на числовых кодах вместо текста), который может быть очень быстро транслирован в родной машинный код. Наличие этого кода с четко определенным универсальным синтаксисом дает ряд существенных преимуществ. 
Это значит, что файл, содержащий инструкции псевдокода, может быть размещен на любой платформе; во время исполнения финальная стадия компиляции может быть легко осуществлена, что позволит выполнить код на конкретной платформе. Другими словами, компилируя в IL, вы получаете платформенную независимость .NET. 

Другим преимуществом подхода является повышение производительности. IL всегда компилируется оперативно (Just-In-Time, или JIT-компиляция). Вместо компиляции всего приложения за один проход (что может привести к задержкам при запуске), JIT-компилятор просто компилирует каждую порцию кода при ее вызове (just-in-time~---~оперативно). Если промежуточный код однажды скомпилирован, то результирующий машинный исполняемый код сохраняется до момента завершения работы приложения, поэтому его перекомпиляция при повторных вызовах не требуется.
 
Компания Microsoft заявляет, что такой процесс более эффективен, чем компиляция всего приложения при запуске, поскольку высока вероятность того, что большие куски кода приложения на самом деле не будут выполняться при каждом запуске. При использовании JIT-компилятора такой код никогда не будет скомпилирован. Это объясняет, почему можно рассчитывать на то, что выполнение родного управляемого кода IL будет почти настолько же быстрым, как и выполнение родного машинного кода. Финальная стадия компиляции проходит во время выполнения, JIT-компилятор на этот момент уже знает, на каком типе процессора будет запущена программа. Это значит, что он может оптимизировать финальный исполняемый код, используя инструкции конкретного машинного кода, предназначенные для конкретного процессора.

Преимущества:
\begin{itemize}
	\item мультипарадигменность;
	\item удобный и понятный синтаксис, поддерживающий огромное количество возможностей;
	\item наличие большой стандартной библиотеки;
	\item возможность интегрировать написанные на C\# модули в системы, написанные на других .NET-совместимых языках;
	\item наличие сборщика мусора;
	\item высокое быстродействие.
\end{itemize}

Недостатки: ограниченное использование во встраиваемых системах.~\cite{WikiCSharp}

\subsubsection{D}
D~---~компилируемый, статически и строго типизированный мултипарадигменный язык с Си-подобным синтаксисом, созданный Уолтером Брайтом из компании Digital Mars. Изначально был задуман как переосмысление языка C++, однако, несмотря на значительное влияние С++, не является его вариантом. В D были заново реализованы некоторые свойства C++, также язык испытал влияние концепций из других языков программирования, таких как Java, Python, Ruby, C\# и Eiffel. При создании языка D была сделана попытка соединить производительность компилируемых языков программирования с безопасностью и выразительностью динамических. Он является языком более высокого уровня, нежели C++. Выделение памяти в языке D полностью контролируется методикой «сборки мусора».

Преимущества:
\begin{itemize}
	\item высокая производительность;
	\item мультипарадигменность;
	\item наличие сборщика мусора;
	\item удобный и понятный синтаксис.
\end{itemize}

Недостатки:
\begin{itemize}
	\item язык только набирает популярность, поэтому число пользователей не так велико, как у других языков;
	\item непродуманная и имеющая ограниченный функционал стандартная библиотека.~\cite{WikiD}
\end{itemize}

\subsubsection{Java}
Java~---~объектно-ориентированный язык программирования, разработанный компанией Sun Microsystems (в последующем приобретённой компанией Oracle). Приложения Java обычно транслируется в специальный байт-код, поэтому они могут работать на любой виртуальной Java-машине (JVM) вне зависимости от компьютерной архитектуры. Дата официального выпуска~---~23 мая 1995 года.

Программы на Java транслируются в байт-код, выполняемый виртуальной машиной Java (JVM) — программой, обрабатывающей байтовый код и передающей инструкции оборудованию как интерпретатор.
Дюк, талисман Java

Достоинством подобного способа выполнения программ является полная независимость байт-кода от операционной системы и оборудования, что позволяет выполнять Java-приложения на любом устройстве, для которого существует соответствующая виртуальная машина. Другой важной особенностью технологии Java является гибкая система безопасности благодаря тому, что исполнение программы полностью контролируется виртуальной машиной. Любые операции, которые превышают установленные полномочия программы (например, попытка несанкционированного доступа к данным или соединения с другим компьютером) вызывают немедленное прерывание.

Разработка приложений с использованием языка Java производится быстрее, чем на С++, так как Java избавлена от низкоуровневых проблем (таких, как, например, выделение и освобождение памяти вручную).

Преимущества:
\begin{itemize}
	\item высокая распространённость языка;
	\item большое количество готовых библиотек и компонентов;
	\item богатая встроенная библиотека.
\end{itemize}

Недостатки:
\begin{itemize}
	\item низкая производительность, связанная с реализацией виртуальной машины (JVM);
	\item ограниченные возможности языка.~\cite{WikiJava}
\end{itemize}

\subsubsection{Python}
Python~---~высокоуровневый язык программирования общего назначения, ориентированный на повышение производительности разработчика и читаемости кода. Синтаксис ядра Python минималистичен. Python поддерживает динамическую типизацию, то есть тип переменной определяется только во время исполнения. В то же время стандартная библиотека включает большой объём полезных функций. Разработка языка Python была начата в конце 1980-х годов сотрудником голландского института CWI Гвидо ван Россумом.

Python поддерживает несколько парадигм программирования, в том числе структурное, объектно-ориентированное, функциональное, императивное и аспектно-ориентированное. Основные архитектурные черты — динамическая типизация, автоматическое управление памятью, полная интроспекция, механизм обработки исключений, поддержка многопоточных вычислений и удобные высокоуровневые структуры данных. Код в Питоне организовывается в функции и классы, которые могут объединяться в модули (они в свою очередь могут быть объединены в пакеты).

Эталонной реализацией Python является интерпретатор CPython, поддерживающий большинство активно используемых платформ. Есть реализации интерпретаторов для JVM (с возможностью компиляции), MSIL (с возможностью компиляции), LLVM и других.

Преимущества:
\begin{itemize}
	\item синтаксис поддерживает большое число функций;
	\item существуют реализации языка для многих популярных платформ и виртуальных машин;
	\item богатая встроенная библиотека.
\end{itemize}

Недостатки:
\begin{itemize}
	\item низкая производительность в интерпретируемых реализациях;
	\item отсутствие концепции будущего развития языка.~\cite{WikiPython}
\end{itemize}

\subsubsection{Ruby}
Ruby~---~динамический, рефлективный, интерпретируемый высокоуровневый язык программирования для быстрого и удобного объектно-ориентированного программирования. Язык обладает независимой от операционной системы реализацией многопоточности, строгой динамической типизацией, сборщиком мусора и многими другими возможностями. Ruby близок по особенностям синтаксиса к языкам Perl и Eiffel, по объектно-ориентированному подходу~---~к Smalltalk. Также некоторые черты языка взяты из Python, Lisp, Dylan и CLU. Создателем языка является Юкихиро Мацумото. Первая версия была опубликована в 1995 году.

Ruby~---~полностью объектно-ориентированный язык. В нём все данные являются объектами, в отличие от многих других языков, где существуют примитивные типы. Каждая функция~---~метод. Ruby является мультипарадигменным языком: он поддерживает процедурный стиль (определение функций и переменных вне классов), объектно-ориентированный (всё~---~объект), функциональный (анонимные функции, замыкания, возврат значения всеми инструкциями, возврат функцией последнего вычисленного значения). Он поддерживает отражение, метапрограммирование, информацию о типах переменных на стадии выполнения.

Для Ruby существуют несколько реализаций: официальный интерпретатор, написанный на Си, JRuby~---~реализация для Java, интерпретатор для платформы .NET IronRuby и др.

Преимущества:
\begin{itemize}
	\item существуют реализации языка для многих популярных платформ и виртуальных машин;
	\item язык имеет лаконичный синтаксис.
\end{itemize}

Недостатки:
\begin{itemize}
	\item низкая производительность в интерпретируемых реализациях;
	\item низкая популярность языка при создании технических систем.~\cite{WikiRuby}
\end{itemize}

\subsubsection{Visual Basic .NET}
Visual Basic .NET~---~объектно-ориентированный язык программирования, реализация языка Visual Basic для платформы .NET.

Преимущества:
\begin{itemize}
	\item простой синтаксис;
	\item возможность интеграции в любые системы на платформе .NET.
\end{itemize}

Недостатки:
\begin{itemize}
	\item ограниченность синтаксиса;
	\item концепция языка является устаревшей и не подвергается серьёзным изменениям.~\cite{WikiVBNet}
\end{itemize}