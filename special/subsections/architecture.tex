\subsection{Архитектура системы}
Архитектура системы проектировалась с учётом возможного расширения и дополнения функционала. Система имеет модульную структуру. Основные функциональные модули:
\begin{itemize}
	\item Thesis.DDMS~---~реализация разработанного метода;
	\item Thesis.Orca~---~реализация алгоритма Orca, описанного в пункте~\ref{subsec:spec:Orca};
	\item Thesis.DataCleansing~---~реализация фильтров, используемых для очищения обучающих выборок от аномалий;
	\item Thesis.Miscellaneous~---~содержит общие классы предметной области, абстракции и функционал, используемый в остальных модулях;
	\item Thesis.App~---~программное приложение (исполняемый файл), связывающее воедино остальные модули и предоставляющее интерфейс пользователя.
\end{itemize}

\subsubsection{Чтение и обработка входных данных}
Реализовано чтение из текстовых и бинарных файлов специального формата. Для обеспечения возможности добавления поддержки других источников и форматов данных представлены абстракции чтения и записи обучающих выборок и синтаксического анализа данных телеметрии.

UML-диаграмма классов, отвечающих за операции над данными, приведена в приложении~\ref{app:UML:Class:DataOperations}. Описание элементов представлено ниже.

\uline{FieldType}~---~тип параметра в векторах входных данных. Возможные значения: \textit{IgnoreFeature} (игнорировать данный параметр при чтении), \textit{Continuous} (непрерывный параметр), \textit{Discrete} (дискретный с фиксированным множеством значений), \textit{DiscreteDataDriven} (дискретный, возможные значения параметра получаются из входных данных).

\uline{Field}~---~класс параметра. Имеет имя, тип, весовой коэффициент, список возможных значений (для дискретных параметров).

\uline{Record}~---~вектор данных (запись). Содержит идентификатор, значения непрерывных и дискретных параметров.

\uline{IRecordParser<T>}~---~абстракция над конвертером, преобразующим единицу входных данных (строку, значения датчиков, подключенных к системе, и т.п.) в вектор (запись) типа \textit{Record}.

\uline{PlainTextParser}~---~реализация конвертера \textit{IRecordParser} для преобразования текстовой строки в вектор.

\uline{DataFormatException}~---~исключение, создающееся при ошибках в текстовых входных данных.

\uline{StringHelper}~---~класс с вспомогательными методами для обработки строк. Позволяет разбить строку на части, используя указанные символы-разделители.

\uline{IDataReader}~---~абстракция над чтением из источника данных. Предоставляет интерфейс для чтения векторов, сброса источника на начальную позицию, получения текущей позиции, описания параметров векторов и флага достижения конца данных.

\uline{IDataWriter}~---~абстракция над записью в источник данных. Предоставляет интерфейс для записи вектора и получения количества уже записанных векторов.

\uline{PlainTextReader}~---~реализация интерфейса \textit{IDataReader} для чтения описания параметров и обучающих выборок из текстовых файлов.

\uline{BinaryDataReader}~---~реализация интерфейса \textit{IDataReader} для чтения описания параметров и обучающих выборок из бинарных файлов специального формата.

\uline{BinaryDataWriter}~---~реализация интерфейса \textit{IDataWriter} для записи описания параметров и обучающих выборок в бинарные файлы специального формата.

\uline{BatchDataReader}~---~реализация интерфейса \textit{IDataReader} для чтения записей блоками по указанному количеству штук. Представляет собой реализацию паттерна декоратор~\cite{GangOfFourDesignPatterns} над объектом типа \textit{IDataReader}.

\uline{IScaling}~---~абстракция над методом нормализации данных. Предоставляет интерфейс для прямого и обратного масштабирования вектора или только его непрерывных параметров.

\uline{MinmaxScaling}~---~реализация интерфейса \textit{IScaling} для минимаксной нормализации по формуле~\eqref{eq:spec:DDMS:MinimaxNormalization}.

\uline{StandardScaling}~---~реализация интерфейса \textit{IScaling} для нормализации с помощью стандартного отклонения по формуле~\eqref{eq:spec:DDMS:StandardNormalization}.

\uline{ScaleDataReader}~---~реализация интерфейса \textit{IDataReader} для нормализации данных при чтении из источника. Представляет собой реализацию паттерна декоратор~\cite{GangOfFourDesignPatterns} над объектом типа \textit{IDataReader}.

\subsubsection{Поиск аномалий в обучающих выборках с помощью алгоритма Orca}
UML-диаграмма классов для поиска аномалий в обучающих выборках приведена в приложении~\ref{app:UML:Class:Orca}. Описание элементов представлено ниже.

\uline{OrcaAD}~---~класс, реализующий алгоритм Orca (см. подраздел~\ref{subsec:spec:Orca}).

\uline{Outlier}~---~структура данных, хранящая информацию об аномалии. Содержит идентификатор записи и значение степени аномальности.

\uline{BinaryShuffle}~---~реализует рандомизацию данных обучающей выборки (требуется для работы алгоритма Orca). Записывает рандомизированные данные в бинарный файл. Рандомизация происходит в $n$ итераций следующим образом: создаётся $m$ временных файлов, в которые случайным образом распределяются элементы обучающей выборки. Затем файлы объединяются в порядке, определяемом с помощью перетасовки Дональда Кнута~\cite{KnuthSeminumAlgo}.

\uline{DataHelper}~---~предоставляет удобный интерфейс для класса BinaryShuffle.

\uline{ScoreFunction}~---~описывает тип функции оценки степени аномальности для алгоритма Orca. Класс \uline{ScoreFunctions} содержит несколько вариантов таких функций.

\uline{BinaryHeap}~---~реализация структуры данных «двоичная куча»~\cite{AlgorithmsCormen}. Используется алгоритмом Orca для определения наиболее удалённых ближайших соседей.

\subsubsection{Очистка обучающих выборок от аномалий}
UML-диаграмма классов, используемых для очистки обучающих выборок от аномалий, приведена в приложении~\ref{app:UML:Class:DataCleansing}. Описание элементов представлено ниже.

\uline{IAnomaliesFilter}~---~абстракция над типом фильтра аномалий. Предоставляет интерфейс для метода \textit{Filter}, принимающего на вход коллекцию объектов типа \textit{Outlier} и возвращающего из неё только те элементы, которые прошли через фильтр (которые являются аномалиями).

\uline{DifferenceFilter}~---~разностный фильтр. Псевдокод алгоритма фильтрации показан в листинге~\ref{lst:spec:Filters:DiffPseudocode}.

\begin{algorithm}[hb!]
\caption{Псевдокод алгоритма фильтрации для разностного фильтра}
\label{lst:spec:Filters:DiffPseudocode}
\begin{algorithmic}[1]
\REQUIRE значения степени аномальности для записей в обучающей выборке, отсортированные по убыванию; $\Delta$, максимальная разность степеней аномальности для двух подряд идущих записей
\ENSURE аномальные записи
\IF{количество записей $< 2$}
	\RETURN $\varnothing$
\ENDIF
\FOR{$i$ от количества записей в выборке до $0$}
	\STATE $\delta \leftarrow$ разность значений степени аномальности ($i-1$)-й и $i$-й записей
	\IF{$\delta > \Delta$}
		\RETURN $i$ записей с самыми большими значениями степени аномальности
	\ENDIF
\ENDFOR
\RETURN $\varnothing$
\end{algorithmic}
\end{algorithm}

\uline{ThresholdFilter}~---~пороговый фильтр. Псевдокод алгоритма фильтрации показан в листинге~\ref{lst:spec:Filters:ThresholdPseudocode}.

\begin{algorithm}[hb!]
\caption{Псевдокод алгоритма фильтрации для порогового фильтра}
\label{lst:spec:Filters:ThresholdPseudocode}
\begin{algorithmic}[1]
\REQUIRE значения степени аномальности для записей в обучающей выборке, отсортированные по убыванию; $\rho$, максимальное значение степени аномальности
\ENSURE аномальные записи
\STATE $\mu \leftarrow$ минимальное значение степени аномальности для всех записей в обучающей выборке
\FORALL{записей в выборке}
	\STATE $x \leftarrow$ значение степени аномальности текущей записи
	\IF{$x - \mu > \rho$}
		\RETURN текущую запись
	\ENDIF
\ENDFOR
\end{algorithmic}
\end{algorithm}

\uline{GaussianFilter}~---~гауссовый фильтр. Псевдокод алгоритма фильтрации показан в листинге~\ref{lst:spec:Filters:GaussianPseudocode}.

\begin{algorithm}[h!]
\caption{Псевдокод алгоритма фильтрации для гауссового фильтра}
\label{lst:spec:Filters:GaussianPseudocode}
\begin{algorithmic}[1]
\REQUIRE значения степени аномальности для записей в обучающей выборке, отсортированные по убыванию
\ENSURE аномальные записи
\STATE $M \leftarrow$ мат. ожидание степени аномальности
\STATE $D \leftarrow$ дисперсия степени аномальности
\STATE $\sigma \leftarrow \sqrt{D}$
\FORALL{записей в выборке}
	\STATE $x \leftarrow$ значение степени аномальности текущей записи
	\IF{$x > M + 3\sigma$}
		\RETURN текущую запись
	\ENDIF
\ENDFOR
\end{algorithmic}
\end{algorithm}

\uline{ScaleDataReader}~---~реализация интерфейса \textit{IDataReader} для пропуска аномальных записей при чтении из источника. Если текущая запись является аномальной, то объект данного класса переходит к следующей записи. Представляет собой реализацию паттерна декоратор~\cite{GangOfFourDesignPatterns} над объектом типа \textit{IDataReader}.

\subsubsection{Построение модели системы}
UML-диаграмма классов для построения модели системы приведена в приложении~\ref{app:UML:Class:DDMS}. Описание элементов представлено ниже.

\uline{DistanceMetric}~---~описывает тип метрики пространства. Класс \uline{DistanceMetrics} содержит несколько вариантов таких метрик, в частности, показанную в формуле~\eqref{eq:spec:DDMS:Distance}, и её квадрат.

\uline{Weights}~---~класс, содержащий весовые коэффициенты параметров.

\uline{Cluster}~---~представляет собой кластер для разработанного метода. Содержит значения непрерывных параметров для верхней и нижней границы и список допустимых значений для каждого дискретного параметра. Позволяет добавлять в себя вектор, расширяя свои границы и обновляя списки допустимых значений так, чтобы добавляемый вектор находился внутри кластера.

\uline{ClusterDistance}~---~описывает тип функции для определения расстояния между вектором и кластером. Класс \uline{ClusterDistances} содержит несколько вариантов таких функций.

\uline{ClusterDatabase}~---~база кластеров. Реализует всю логику по созданию новых кластеров и проверке попадания векторов в существующие.

\uline{Regime}~---~представляет режим работы системы. Содержит имя и базу кластеров. Реализует логику проверки принадлежности вектора режиму и расчёта расстояния до него.

\uline{SystemModel}~---~представляет модель системы. Содержит в себе список режимов работы, реализует логику по добавлению режимов и определению ближайшего к входному вектору режима.

\FloatBarrier