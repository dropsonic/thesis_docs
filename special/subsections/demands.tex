\subsection{Предъявляемые требования}
\subsubsection{Поддержка объектно-ориентированной парадигмы}
Объектно-ориентированне программирование (ООП)~---~парадигма программирования, в которой основными концепциями являются понятия объектов и классов. В случае языков с прототипированием вместо классов используются объекты-прототипы. В центре ООП находится понятие объекта. Объект~---~это сущность, которой можно посылать сообщения и которая может на них реагировать, используя свои данные. Объект~---~это экземпляр класса. Данные объекта скрыты от остальной программы (инкапсулированы).
Наличие инкапсуляции достаточно для объектности языка программирования, но ещё не означает его объектной ориентированности~---~для этого требуется наличие наследования.
Но даже наличие инкапсуляции и наследования не делает язык программирования в полной мере объектным с точки зрения ООП. Основные преимущества ООП проявляются только в том случае, когда в языке программирования реализован полиморфизм; то есть возможность объектов с одинаковой спецификацией иметь различную реализацию.

Основные концепции и понятия ООП представлены ниже.

\uline{Абстрагирование}~---~это способ выделить набор значимых характеристик объекта, исключая из рассмотрения незначимые. Соответственно, абстракция — это набор всех таких характеристик.

\uline{Инкапсуляция}~---~это свойство системы, позволяющее объединить данные и методы, работающие с ними в классе, и скрыть детали реализации от пользователя.

\uline{Наследование}~---~это свойство системы, позволяющее описать новый класс на основе уже существующего с частично или полностью заимствующейся функциональностью. Класс, от которого производится наследование, называется базовым, родительским или суперклассом; новый класс~---~потомком, наследником или производным классом.

\uline{Полиморфизм}~---~это свойство системы использовать объекты с одинаковым интерфейсом без информации о типе и внутренней структуре объекта.

\uline{Класс}~---~описываемая на языке терминологии (пространства имён) исходного кода модель ещё не существующей сущности (объекта). Фактически он описывает устройство объекта, являясь своего рода чертежом. Говорят, что объект~---~это экземпляр класса. При этом в некоторых исполняющих системах класс также может представляться некоторым объектом при выполнении программы посредством динамической идентификации типа данных. Обычно классы разрабатывают таким образом, чтобы их объекты соответствовали объектам предметной области.

\uline{Объект}~---~сущность в адресном пространстве вычислительной системы, появляющаяся при создании экземпляра класса или копирования прототипа (например, после запуска результатов компиляции и связывания исходного кода на выполнение).

\uline{Прототип}~---~объект-образец, по образу и подобию которого создаются другие объекты. Объекты-копии могут сохранять связь с родительским объектом, автоматически наследуя изменения в прототипе; эта особенность определяется в рамках конкретного языка.~\cite{PaisonOOP}

Основными принципы объектно-ориентированного проектирования представлены ниже. Данный набор принципов принято обозначать аббревиатурой \textit{SOLID}.

\uline{Принцип единственной обязанности} (Single Responsibility Principle): каждый объект должен иметь одну обязанность и эта обязанность должна быть полностью инкапсулирована в класс. Все его сервисы должны быть направлены исключительно на обеспечение этой обязанности. Таким образом, класс или модуль должны иметь одну и только одну причину измениться.

\uline{Принцип открытости/закрытости} (Open/Closed Principle): программные сущности (классы, модули, функции и т.п.) должны быть открыты для расширения, но закрыты для изменения. Это означает, что такие сущности могут позволять менять свое поведение без изменения их исходного кода.

\uline{Принцип подстановки Барбары Лисков} (Liskov Substitution Principle): объекты в программе могут быть заменены их наследниками без изменения свойств программы. Функции, которые используют базовый тип, должны иметь возможность использовать подтипы (наследники) базового типа, не зная об этом.

\uline{Принцип разделения интерфейса} (Interface Segregation Principle): клиенты не должны зависеть от методов, которые они не используют. Интерфейсы, содержащие слишком много полей и методов, необходимо разделять на более маленькие и специфические, чтобы клиенты маленьких интерфейсов знали только о методах, которые необходимы им в работе. В итоге, при изменении метода интерфейса не должны меняться клиенты, которые этот метод не используют.

\uline{Принцип инверсии зависимостей} (Dependency Inversion Principle): зависимости внутри системы строятся на основе абстракций. Модули верхнего уровня не зависят от модулей нижнего уровня. Абстракции не должны зависеть от деталей. Детали должны зависеть от абстракций.~\cite{RMartinAgile}

Применение объектно-ориентированной парадигмы позволяет улучшить понимание исходного кода, упростить его дальнейшую поддержку, использование и усовершенствование, так как модули и классы в коде представляют собой проекции реальных объектов предметной области. В частности, в проектируемой системе такими объектами являются модель системы, режимы работы, кластеры, входные векторы и т.д. Поэтому требуется, чтобы язык программирования в полной мере поддерживал все концепции ООП.

\subsubsection{Удобство использования}
Под удобством использования подразумевается как логичность, приспособленность конструкций конкретного языка к поставленным задачам, так и удобство средств разработки (уже независимо от выбранного языка). Кроме того, следует учесть качество стандартной библиотеки и количество готовых компонентов (фреймворков, библиотек и т.п.), которые возможно использовать для решения типовых задач, программируя на данном языке. Удобство использования языка непосредственно влияет на время, затрачиваемое на написание программ на нём.

\subsubsection{Возможность повторного использования кода}
Повторное использование кода~---~методология проектирования компьютерных и других систем, заключающаяся в том, что система (компьютерная программа, программный модуль) частично либо полностью должна составляться из частей, написанных ранее компонентов и/или частей другой системы, и эти компоненты должны применяться более одного раза (если не в рамках одного проекта, то хотя бы разных). Повторное использование~---~основная методология, которая применяется для сокращения трудозатрат при разработке сложных систем.

Самый распространённый случай повторного использования кода~---~библиотеки программ. Библиотеки предоставляют общую достаточно универсальную функциональность, покрывающую избранную предметную область.

Данная система разрабатывается, как готовый программный продукт, однако должна быть возможность встраивания элементов системы (в частности, реализации метода диагностики аномалий) в другие программные продукты и системы. Также должна быть возможность доработки элементов системы другими специалистами, для которых специализированный или редко используемый язык может создать затруднения и увеличить затраты на использование модулей данной системы. Поэтому язык должен быть достаточно популярным в профессиональной среде и предоставлять средства для удобного создания модулей и отдельных библиотек.

\subsubsection{Быстродействие}
В зависимости от языка программирования быстродействие реализаций одного и того же алгоритма может значительно отличаться. Несмотря на то, что систему не предполагается использовать во встраиваемых ЭВМ, а ПЭВМ могут обеспечить мощности, более чем достаточные для подобной задачи, данный критерий всё равно представляется важным, так как система должна иметь возможность работать в режиме реального времени.