\spchapter{Заключение}
В рамках дипломной работы были выполнены следующие пункты:
\begin{itemize} 
	\item произведён анализ существующих алгоритмов выявления аномалий без учителя, выделены их достоинства и недостатки;
	\item создан метод мониторинга состояния ЛА на основе алгоритмов интеллектуального анализа данных, устраняющий недостатки существующих методов;
	\item разработана его программная реализация.
\end{itemize}

В результате разработки все требования задания были полностью удовлетворены. Разработанная программная реализация метода позволет строить модель произвольной системы для последующего контроля её состояния только на основе данных телеметрии при различных режимах её работы. Метод не требует априорных знаний о предметной области, назначении системы, её составе, конструкции. Программная реализация метода обладает способностью обучаться на больших массивах исходных данных, учитывает как дискретные, так и непрерывные параметры, устойчива к аномалиями в обучающих выборках и может осуществлять контроль даже при отсутствии некоторых параметров в измерениях. Разработанное ПО обладает достаточным быстродействием для работы в режиме реального времени.

Результаты проведённого тестирования свидетельствуют об отсутствии ошибок контроля в работе системы и показывают её соответствие заявленным характеристикам.

Система может применяться для своевременного контроля состояния ЛА научно-измерительными пунктами (НИП) на основе данных телеметрии. Разработанный метод может быть реализован в качестве бортовой системы мониторинга, но это исследование выходит за рамки данной дипломной работы.

В экономической части дипломной работы была проведена оценка экономической эффективности разработанного ПО, которая показала, что внедрение разработанной системы в эксплуатацию является целесообразным.

В разделе «Охрана труда и окружающей среды» был проведен анализ условий труда при разработке программного обеспечения. Также был рассчитан уровень шума в помещении, позволяющий обеспечить комфортные условия труда и повысить его производительность, а также даны рекомендации по обеспечению условий труда в соответствии с принятыми нормами и ГОСТами.