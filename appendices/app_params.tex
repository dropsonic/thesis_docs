\chapter{Параметры запуска программы}
\label{app:AppOptions}

\begin{longtable}[H]{|p{188pt}|p{195pt}|C{90pt}|}
\caption{Параметры запуска программы}
\label{tab:spec:AppOptions}
\\ \hline
Формат & Описание & Обязательный \\ \hline
\endfirsthead
\LTcontcaption{tab:spec:AppOptions}
\\ \hline
Формат & Описание & Обязательный \\ \hline
\endhead

имя\_файла & Файл с описанием формата параметров векторов. & Да \\
\hline
-r имя\_файла1\dots \ имя\_файлаN & Файлы с обучающими выборками для номинальных режимов работы системы & Да \\
\hline
-a имя\_файла1\dots \ имя\_файлаN & Файлы с обучающими выборками для внештатных (аномальных) режимов работы системы & Нет \\
\hline
-f тип\_фильтра [параметр\_фильтра] & Тип фильтра, использующегося при удалении аномалий из обучающих выборок. Допустимые значения: gaussian (гауссовый), difference (разностный), threshold (пороговый). Для разностного фильтра необходимо указать максимальную разность, для пороговго - величину порога. Значение по умолчанию: gaussian. & Нет \\
\hline
-o имя\_файла & Имя файла для сохранения вывода программы. & Нет \\
\hline
-n вид\_нормализации & Вид нормализации (масштабирования) векторов с данными. Допустимые значения: minimax (минимаксная), standard (по стандартному отклонению). Значение по умолчанию: minimax. & Нет \\
\hline
-m тип\_метрики & Тип метрики пространства. Допустимые значения: euclid (евклидова), sqreuclid (квадрат евклидовой). Значение по умолчанию: euclid. & Нет \\
\hline
-d тип\_расстояния & Тип функции для измерения расстояния между кластером и вектором. Допустимые значения: kmeans (до центра кластера), nearest (до ближайшей точки кластера). Значение по умолчанию: nearest. & Нет \\
\hline
-v строка & Строка, замещающая отсутствующие значения в векторах в файлах с обучающими выборками. Значение по умолчанию: «?». & Нет \\
\hline
\end{longtable}