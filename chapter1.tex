\chapter{Специальная часть}
\section{Постановка задачи}
Разработать метод мониторинга состояния ЛА на основе методов интеллектуального анализа данных. Реализовать программную систему, использующую данный метод.

Система должна удовлетворять следующим требованиям:
\begin{itemize}
	\item строить модель системы только на основе телеметрии при различных режимах её работы, без априорных данных о её назначении, составе, конструкции (обучение без учителя);
	\item обладать способностью классифицировать аномалии в работе системы;
	\item обрабатывать большие массивы входных данных (несколько десятков тысяч записей) за конечное время;
	\item определять состояние системы в режиме реального времени.
\end{itemize}

%------------------------------------------------------------------
\section{Анализ существующих методов выявления аномалий без учителя}
На данный момент существует несколько методов, для которых доказана возможность применения их в системах контроля и диагностики ЛА. Такими методами являются Orca, IMS (Inductive Monitoring System), GritBot, GMM (Gaussian Mixture Model), LDS (Linear Dynamic System) и One-Class SVM (Support Vector Machine)~\cite{MartinCompUnsupervisedDetectionMethods}.

\subsection{Orca}
Orca~---~метод поиска аномалий без учителя, использующий подход «ближайшего соседа» (nearest neighbor) для поиска аномалий~\cite{SchwabacherMachLearnAppl}.