\chapter{Специальная часть}
\section{Постановка задачи}
Разработать метод мониторинга состояния ЛА на основе методов интеллектуального анализа данных. Реализовать программную систему, использующую данный метод.

Система должна удовлетворять следующим требованиям:
\begin{itemize}
	\item строить модель системы только на основе телеметрии при различных режимах её работы, без априорных данных о её назначении, составе, конструкции (обучение без учителя);
	\item обладать способностью классифицировать аномалии в работе системы;
	\item обрабатывать большие массивы входных данных (несколько десятков тысяч точек) за конечное время;
	\item определять состояние системы в режиме реального времени.
\end{itemize}

%------------------------------------------------------------------
\section{Анализ существующих методов выявления аномалий без учителя}
На данный момент существует несколько методов, для которых доказана возможность применения их в системах контроля и диагностики ЛА. Такими методами являются Orca, IMS (Inductive Monitoring System), GritBot, GMM (Gaussian Mixture Model), LDS (Linear Dynamic System) и One-Class SVM (Support Vector Machine)~\cite{MartinCompUnsupervisedDetectionMethods}.

\subsection{Orca}
Orca~---~метод поиска аномалий без учителя, использующий подход «ближайшего соседа» (nearest neighbor) для поиска аномалий~\cite{SchwabacherMachLearnAppl}. Orca относится к методам обнаружения аномалий, основанных на измерении расстояний между точками (distance-based).

Понятие аномалии для данного класса методов определено следующим образом: «объект $O$ в выборке $T$ является аномалией, если по крайней мере доля $p$ из всех объектов в $T$ лежит дальше от $O$, чем расстояние $D$»~\cite{KnorrNgDistBasedAlgorithms}. Distance-based методы являются обобщением некоторых статистических тестов на аномальность. Кнорр и Нг предложили простейший алгоритм на вложенных циклах (Nested Loop, NL)~\cite{KnorrNgDistBasedAlgorithms}, который находит аномалии путём вычисления расстояния между всеми точками в исходной выборке. Сложность данного алгоритма составляет $O(kN^2)$, где $k$ --- размерность пространства, а $N$ --- размер выборки.

Несмотря на то, что были разработаны более эффективные c т.з. вычислительной сложности алгоритмы (\cite{TaoMiningDistBasedOutliersFromLargeDB}~и~\cite{AngiulliVeryEfficientMiningDistBasedOutliers}), на практике наиболее сложным является определение расстояния $D$, по достижению которого точку следует считать аномалией. Может потребоваться непредсказуемо большое число итераций, чтобы найти подходящее значение $D$. Найти интервал $[D_{min}, D_{max}]$ возможно путём полного перебора, как показано в~\cite{RamaswamyEffAlgoMiningOutliers}, но данный подход обладает слишком высокой вычислительной сложностью.

\subsection{GritBot}
К основным недостаткам относится тот факт, что данный метод загружает весь массив исходных данных в память; таким образом, невозможно обрабатывать сколь либо большие выборки.